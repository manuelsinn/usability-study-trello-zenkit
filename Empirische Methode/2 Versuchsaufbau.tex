\subsection{Versuchsaufbau}
Wir entschieden uns für ein within-subjects design, um den besten Kosten-Nutzen Effekt zu erzielen und eine bessere Vergleichbarkeit der Versuchsergebnisse zu erreichen. Um Positions- und Reihenfolgeeffekte auszuschließen nutzten wir vollständiges randomisiertes Ausbalancieren: Ein Zufallsgenerator bestimmte für jede Versuchsperson, mit welcher Ausprägung der unabhängigen Variable, d.h. mit welchem der beiden Systeme die Person starten würde.

Neben dem Versuchsleiter, der die Versuchsperson durch die Studie führte und instruierte, war ein Protokollant präsent um die Geschehnisse zu notieren und einige der abhängigen Variablen aufzuzeichnen. 