\subsection{Operationalisierung der Usabilitykritierien}

Um die Usability der beiden Systeme vergleichen zu können wurden die drei Leitkriterien nach der Norm ISO 9241-11 (\cite{iso}) durch jeweils zwei Maße operationalisiert, um eine höhere Konstruktvalidität und Objektivität zu erreichen.\\


\subsubsection{Effektivität}
Als Maß für die Effektivität wurde für jeden Probanden der Anteil gelöster Teilaufgaben erhoben, sowie die Anzahl an Zeitüberschreitungen. 

Der Anteil gelöster Teilaufgaben beschreibt direkt, wie gut die Versuchspersonen in der Lage waren, das System effektiv zu nutzen. Die Anzahl Zeitüberschreitungen wurde aufgrund der Annahme gewählt, dass eine Zeitüberschreitung in der Studiensituation dem Nichterfüllen der Aufgabe im realen Nutzungskontext entspricht. Das liegt daran, dass die Versuchspersonen im Rahmen der Studie eher angehalten sind, die von ihnen verlangte Aufgabe auch dann zu realisieren wenn es verhältnismäßig lange dauert, während Geduld und Dringlichkeit in der Realität geringer ausfallen würden, und die Aufgabe wahrscheinlich früher abgebrochen werden würde.




\subsubsection{Effizienz} 
Als Maß für die Effizienz wurde die durchschnittliche Zeit zum Lösen einer Aufgabe erhoben, sowie die Anstrengung durch den NASA TLX Fragebogen nach \cite{hart1988development}.

Die durchschnittliche Zeit zum Lösen einer Aufgabe beschreibt direkt die Effizienz der Handhabung des Systems, da die Schnelligkeit einen grundlegenden Aspekt der Effizienz darstellt. Hierbei wurde eine Aufgabe dann als gelöst gewertet, wenn mindestens zwei von drei Unteraufgaben gelöst wurden. War dies nicht der Fall, und/oder die Zeitobergrenze wurde erreicht, so wurde für diese Aufgabe die Maximalzeit angenommen.

Unter anderem  hat eine Studie von \cite{galy2018measuring} die Möglichkeit der einzelnen Nutzung der Dimensionen des NASA TLX untersucht, bei der die mutmaßlichen Beziehungen der Kategorien mentaler Belastung aufgestellt wurden. Die Dimension 'Anstrengung' des NASA TLX wurde gewählt, da diese nach \cite{galy2018measuring} die Einschätzung der kognitiven Ressourcen widerspiegelt, und dabei sowohl die Komplexität der Situation (mit den Dimension der geistigen, physischen und zeitlichen Anforderung), sowie die Erregung (mit der Dimension der Frustration) miteinbezieht.




\subsubsection{Zufriedenstellung} 
Als Maß für die Zufriedenstellung der Nutzer wurde die Attraktivitätsskala des AttrakDiff Fragebogens nach \cite{hassenzahl2003attrakdiff} erhoben, sowie eine Gesamt-User-Experience-Kennzahl (nachfolgend UX-Wert).

Die Attraktivitätsskala des AttrakDiff Fragebogens wurde genutzt, da diese sowohl von den hedonischen als auch von den pragmatischen Qualitäten eines Produktes zu gleichen Teilen beeinflusst wird (\cite{hassenzahl2003attrakdiff}). In einem weiteren Artikel von \cite{hassenzahl2008user} wird das Attraktivitätsmerkmal beschrieben als die "Globale positiv-negativ Bewertung des Produkts", was für uns eine gute Annäherung an die insgesamte Zufriedenheit mit dem Produkt darstellt.

Der UX-Wert berechnet sich aus dem Verhältnis der während der Explorationsphase gesetzten Marker mit positiver bzw. negativer Valenz: Die Anzahl positiver Marker abzüglich der Anzahl negativer Marker wird geteilt durch die Gesamtanzahl. Somit wird sowohl die Qualität (negativ oder positiv) als auch die Quantität (die Anzahl der Marker) bezüglich des Systems miteinbezogen, die für eine Versuchsperson überwogen hat. Unter der Annahme, dass eine Versuchsperson mit einem System z.B. weniger zufrieden ist wenn sie mehr negative als positive Aspekte benennt, ist der UX-Wert ein valides Maß zur Einschätzung der Zufriedenheit.