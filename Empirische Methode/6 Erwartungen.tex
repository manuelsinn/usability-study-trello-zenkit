\subsection{Erwartungen}

Zu Beginn des Projektes bestand unsere Einschätzung in einer deutlichen Überlegenheit Trellos. Die dieser Einschätzung zum Teil widersprechenden Ergebnisse der analytischen Methode haben unsere Erwartungshaltung eingeschränkt und relativiert. Wie im Ergebnisteil der Heuristischen Evaluation zu erkennen, blieb der Eindruck jedoch weiterhin bestehen, dass Trello das System mit der besseren Usability sei, nicht zuletzt aufgrund der Verteilung der Probleme auf die Heuristiken, welche ein Indiz für Zenkits unausgereiftere Nutzeroberfläche darstellte. Somit formulierten wir unsere Hypothesen weiterhin gerichtet, mit der Erwartung, dass Trello eine bessere Usability aufweisen würde.

\begin{outline}[enumerate]
\1 Effektivität
    \2 Wenn Trello eine bessere Effektivität hat als Zenkit, dann hat Trello einen höheren Anteil erfüllter Teil-Aufgaben als Zenkit.
    \2 Wenn Trello eine bessere Effektivität hat als Zenkit,
dann hat Trello weniger Zeitüberschreitungen als Zenkit.

\1 Effizienz
    \2 Wenn Trello eine bessere Effizienz hat als Zenkit, dann brauchen die Probanden für das Lösen der Aufgaben bei Trello weniger Zeit als bei Zenkit.
    \2 Wenn Trello eine bessere Effizienz hat als Zenkit,
dann messen wir bei Trello eine niedrigeren durchschnittlichen NASA-TLX-Anstrengungs-Score als bei Zenkit.


\1 Zufriedenheit
    \2 Wenn Trello eine höhere Zufriedenheit hat als Zenkit,
dann erhält Trello  auf der Attraktivitätsskala des AttrakDiff höhere Bewertungen als Zenkit.
    \2 Wenn Trello eine höhere Zufriedenheit hat als Zenkit,
dann messen wir bei Trello einen höheren UX-Wert als bei Zenkit.

\end{outline} 