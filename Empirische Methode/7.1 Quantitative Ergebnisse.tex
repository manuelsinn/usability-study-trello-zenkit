\subsection{Quantitative Ergebnisse}

Die statistische Vorraussetzung des abhängigen t-Tests, die Normalverteilung der Differenzen, wurde jeweils mithilfe eines Shapiro-Wilk Tests auf Normalverteilung mit einem Signifikanzniveau von $\alpha$~=~.05 überprüft. Dieser Test wurde gewählt, da er für kleinere Stichproben nach \cite{power} im Vergleich mit ähnlichen Tests (wie dem Kolmogorov-Smirnov Test) bei einer Monte-Carlo Simulation die beste Teststärke aufweisen konnte.

Wenn der Shapiro-Wilk Test auf Normalverteilung nicht signifikant wurde (und der Datensatz damit als normalverteilt angesehen werden konnte), dann wurde im Anschluss ein abhängiger t-Test gerechnet. Andernfalls wurde ein Wilcoxon Rangsummentest gerechnet, der die Annahme der Normalverteilung nicht vorraussetzt. 

Da mehrere Tests an der gleichen Stichprobe zu rechnen waren, sollte Alpha-Adjustierung, z.B. nach Bonferroni-Holm, angewandt werden. Da keiner der Tests selbst auf dem Standardsignifikanzniveau von .05 signifikant wurde, ist dies jedoch nicht relevant.

Obwohl zwei Variablen nicht als normalverteilt angenommen werden können wird stets Mittelwert und Standardabweichung angegeben (anstatt Median und Interquartilsabstand), da für alle Variablen Intervallskaliertheit angenommen wird.\\





\subsubsection{Effektivität}
Ein Shapiro-Wilk Test zeigte, dass die Differenzen der Anzahl der gelösten Teilaufgaben nicht normalverteilt sind, \textit{W}(27)~=~.918, \textit{p}~<~.05.
Ein Wilcoxon Rangsummentest betreffend der durchschnittlichen Anzahl gelöster Teilaufgaben lieferte keinen signifikanten Unterschied zwischen Trello (\textit{MW}~=~11.714, \textit{SD}~=~2.594) und Zenkit (\textit{MW}~=~11.357, \textit{SD}~=~2.725), \textit{z}~=~1.4, \textit{p}~=~.081, \textit{r}~=~.265.

Ein Shapiro-Wilk Test zeigte, dass die Differenzen der Anzahl der Zeitüberschreitungen nicht normalverteilt sind, \textit{W}(27)~=~.887, \textit{p}~<~.01.
Ein Wilcoxon Rangsummentest betreffend der durchschnittlichen Anzahl der Zeitüberschreitungen lieferte keinen signifikanten Unterschied zwischen Trello (\textit{MW}~=~.464, \textit{SD}~=~.576) und Zenkit (\textit{MW}~=~.536, \textit{SD}~=~.693), \textit{z}~=~-.39, \textit{p}~=~.349, \textit{r}~=~.074.




\subsubsection{Effizienz}
Ein Shapiro-Wilk Test zeigte, dass die Differenzen der Zeiten normalverteilt sind, \textit{W}(27)~=~.983, \textit{p}~=~.922. 
Ein t-Test für abhängige Stichproben betreffend der durchschnittlich benötigten Zeit zur Lösung einer Aufgabe in Sekunden lieferte keinen signifikanten Unterschied zwischen Trello (\textit{MW}~=~73.964, \textit{SD}~=~21.555) und Zenkit (\textit{MW}~=~78.857, \textit{SD}~=~20.348),\\\textit{t}(27)~=~-~1.166, \textit{p}~=~.127, \textit{d}~=~.220.

Ein Shapiro-Wilk Test zeigte, dass die Differenzen der Anstrengungswerte des NASA TLX normalverteilt sind, \textit{W}(27)~=~.944, \textit{p}~=~.141. 
Ein t-Test für abhängige Stichproben betreffend der NASA TLX Werte lieferte keinen signifikanten Unterschied zwischen Trello (\textit{MW}~=~6.786, \textit{SD}~=~4.122) und Zenkit (\textit{MW}~=~8.250, \textit{SD}~=~4.088), \textit{t}(27)~=~-~1.662, \textit{p}~=~.054, \textit{d}~=~.314.




\subsubsection{Zufriedenheit}   
Ein Shapiro-Wilk Test zeigte, dass die Differenzen der UX-Werte normalverteilt sind, \textit{W}(25)~=~.968, \textit{p}~=~.577. 
Ein t-Test für abhängige Stichproben betreffend der durchschnittlichen UX-Werte lieferte keinen signifikanten Unterschied zwischen Trello (\textit{MW}~=~.327, \textit{SD}~=~.649) und Zenkit (\textit{MW}~=~.089, \textit{SD}~=~.661), \textit{t}(25)~=~1.385,
 \textit{p}~=~.089, \textit{d}~=~.272.

Bei der Auswertung des UX-Wertes kam es zum Wegfall zweier Probanden, da diese weder positive noch negative Marker gesetzt hatten. Da hierbei möglicherweise die Aufgabenstellung missverstanden wurde und kein UX-Wert errechnet werden konnte, wurden diese Datenpunkte als ungültig angesehen und ausgeschlossen.

Ein Shapiro-Wilk Test zeigte, dass die Differenzen der Attraktivitätswerte des AttrakDiff nicht normalverteilt sind, \textit{W}(27)~=~.810, \textit{p}~<~.001.
Ein Wilcoxon Rangsummentest betreffend der Attraktivitätswerte lieferte keinen signifikanten Unterschied zwischen Trello (\textit{MW}~=~3.529, \textit{SD}~=~.878) und Zenkit (\textit{MW}~=~3.479, \textit{SD}~=~.750), \textit{W}(27)~=~.262, \textit{p}~=~.398, \textit{d}~=~.050.