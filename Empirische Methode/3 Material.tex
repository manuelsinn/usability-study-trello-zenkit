\subsection{Material}

\subsubsection{Boards}
Zum Testen der beiden Systeme erstellten wir zwei verschiedene Board-Templates mit grundlegenden Szenarien, in die sich die Versuchsperson hineinversetzen sollte (siehe Abbildungen \ref{fig:trelloeinkauf}, \ref{fig:zenkiteinkauf}, \ref{fig:trellogebu}, \ref{fig:zenkitgebu} im Anhang).

Im Ersten der beiden diente das System als eine Art Einkaufliste mit den drei Kategorien "Noch genug vorhanden", "Muss gekauft werden" und "Im Einkaufskorb". Der Fokus bei der Erstellung dieses Szenarios lag bei der Heranführung der Versuchsperson an das System, sodass sie die grundlegenden Funktionen wie Swimlanes\footnote{Eventuell besser bekannt als Listen, die vertikale Unterteilung der Karten}, Karten, oder auch die Drag-and-Drop Interaktion einfach erfahren konnte. Die einfache und übersichtliche Gestaltung mit nur drei Swimlanes und wenigen Karten half dabei, da sie die Versuchsperson nicht unnötig beanspruchte oder überwältigte.

Das zweite Szenario spiegelte die Organisation einer Geburtstagsfeier wider. Hierbei lag der Fokus vor allem auf den zu bewältigenden Aufgaben.



\subsubsection{Aufgaben}
Bei der Erstellung der Aufgaben, welche die Versuchsperson bewältigen sollte, wurde besonders darauf geachtet möglichst viele der gebotenen Funktionen einzuarbeiten: Von Grundfunktionen wie dem Erstellen neuer Karten bis hin zu Expertenfunktionen wie der Wiederherstellung einzelner Karten aus dem Archiv. Das Geburtstagsszenario wurde mit dem Zweck gewählt, die Arbeit mit dem System realitätsnäher zu gestalten, und eine gemeinsame Situation zu bieten, in die sich die Versuchspersonen leicht hineinversetzen konnten.

