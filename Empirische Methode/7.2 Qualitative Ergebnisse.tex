\subsection{Qualitative Ergebnisse}
Die im Interview am Ende der Studie entstandenen Aussagen der Probanden über die gesetzten Marker wurden analysiert und für bessere Vergleichbarkeit in grundlegendere Kategorien eingeteilt. Im Fokus stehen hier diejenigen Kategorien, die für das jeweilige System von mehr als einem Viertel der Stichprobe angeführt wurden, d.h. mehr als sieben mal auftraten (Die restliche Aufschlüsselung siehe Anhang).\\

\subsubsection{Trello}
Die vier meist gelobten Kategorien bei Trello waren
\begin{seriate}
    \item Intuitive Benutzung (\textit{n}~=~20, z.B. VP 7: 'Das Umbenennen der Karten ist schnell und einfach', VP 18: 'Man sieht auf den ersten Blick was man machen kann'),
    \item Minimalistisches Design (\textit{n}~=~17, z.B. VP 9: 'Man bekommt nicht zu viele Infos auf einmal'),
    \item Ästhetik (\textit{n}~=~14, z.B. VP 12: 'Das Design vermittelt ein Gemeinschaftsgefühl') und 
    \item Anpassbarkeit (\textit{n}~=~10, z.B. VP 19: 'Schön und anpassbar, man kann alles [ändern] aber man muss nichts [ändern]').
\end{seriate}{}



In der Kritik stand v.a. 
\begin{seriate}
    \item Fehlende Erwartungskonformität (\textit{n}~=~13, z.B. VP 19: 'Warum habe ich kein Feedback beim Hinzufügen von Mitgliedern bekommen?'), 
    \item das Nichtauffinden von Funktionen (\textit{n}~=~10, z.B. VP 14: 'Ich hatte keine Ahnung wie ich zum Archiv kommen sollte'), 
    \item Unverständliche Bedienelemente (\textit{n}~=~9, z.B. VP 28 über die Butler-Funktion\footnote{Eine Experten-Funktion zur Automatisierung von Aufgaben und deren Eigenschaften, die oft auf Unverständnis stieß.}: 'Was ist das?').
\end{seriate}




\subsubsection{Zenkit}
Bei Zenkit wurden die selben vier Kategorien wie bei Trello am meisten gelobt, jedoch in unterschiedlicher Verteilung und Rangreihenfolge: 
\begin{seriate}
    \item Intuitive Benutzung (\textit{n}~=~18, z.B. VP 16: 'Hier [war] klar, ich kann die Karten nehmen und verschieben.”), 
    \item Ästhetik (\textit{n}~=~13, z.B. VP 25: 'Das Design ist schön 'rund'' oder VP~28: '[Zenkit] sieht insgesamt schöner aus, nicht nur grau, sondern hell und freundlich, moderner.'), 
    \item Anpassbarkeit (\textit{n}~=~8, z.B. VP 16: 'Ich konnte einfach die Labels ändern um es meinen Wünschen anzupassen') und
    \item Minimalistisches Design (\textit{n}~=8, z.B. VP 6: 'Zenkit hat einfach ein schlichtes, übersichtliches Design'). 
\end{seriate}



Auf Seiten der Kritik ähneln sich die beiden Systeme genauso, auch hier unterscheidet sich ausschließlich die Reihenfolge der Kategorien:
\begin{seriate}
    \item Unverständliche Bedienelemente (\textit{n}~=~13, z.B. VP 28: 'Was bedeutet Kanban bzw. Stage?'),
    \item das Nichtauffinden von Funktionen (\textit{n}~=~11, z.B. VP 25 über das Kartenmenü: 'Ich fand es nicht eindeutig, dass man in diesem Fenster den Namen [der Karte] ändern kann') und
    \item Fehlende Erwartungskonformität (\textit{n}~=~8, z.B. VP 14: 'Warum kann man nicht auch Listen archivieren?'). 
\end{seriate}{}