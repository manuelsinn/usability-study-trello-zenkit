\subsection{Versuchsdurchführung}
Nach einer kurzen Begrüßung stellte der Versuchsleiter sich selbst, den Protokollanten und das Projekt vor. Anschließend erhielt die Versuchsperson einen Ablaufzettel und füllte den Vorfragebogen und die Datenschutzerklärung aus.


\subsubsection{Valenzmethode}
Als nächstes erhielt die Versuchsperson die Aufforderung, für zwei Minuten frei das Einkaufslisten-Board zu explorieren. Dabei sollten aufkommende Gefühle beachtet werden, und jedes Mal wenn ein Aspekt des Systems positiv oder negativ auffällt der dazugehörige Marker gesetzt werden. Das Setzen der Marker wurde durch Drücken der Tasten F2 und F3 realisiert, auf denen jeweils ein Zettel mit einem grünen Pluszeichen bzw. einem roten Minuszeichen klebte. Auf das Drücken der Taste folgte eine kurze Feedback-Nachricht auf dem Display zur Bestätigung des Setze des Markers. Um die Erinnerung der Versuchsperson über die gesetzten Marker für das spätere Interview zu unterstützen, wurde zudem ein Blatt Papier für Notizen bereitgestellt. Nach Ablauf der zwei Minuten durchlief die Versuchsperson die selbe Prozedur noch einmal für das zweite System.



\subsubsection{Lösen der Aufgaben}
Nach Abschluss der Valenzmethode begann der Hauptteil der Studie, das Lösen der Aufgaben und das Ausfüllen der anschließenden Fragebögen. Die Versuchsperson wurde dazu angehalten, zunächst den Text der aktuellen Aufgabe zu lesen und sich bei Fragen an den Versuchsleiter zu wenden. Anschließend sollte sie den Beginn ihrer Bearbeitung der Aufgabe laut ansagen, die Aufgabe bearbeiten und das Beenden der Bearbeitung wiederum laut ansagen. Der Protokollant notierte währenddessen sowohl die Bearbeitungszeit für jede Aufgabe, als auch die Erfüllung bzw. Nichterfüllung der einzelnen Unteraufgaben.

Für die ersten vier Aufgaben wurde eine Zeitspanne von zwei Minuten gewährt, für die fünfte Aufgabe drei Minuten, da diese im Vorfeld aufgrund der komplexen Menüführung als schwieriger eingestuft wurde. Sobald die Versuchsperson diese Zeit überschritt, wurde ihr das durch den Versuchsleiter mitgeteilt und zur nächsten Aufgabe gesprungen.

Nach Beendigung aller Aufgaben füllte die Versuchsperson die Fragebögen aus, bevor sie auch diesen Teil der Studie anschließend noch einmal mit dem zweiten System durchlief.



\subsubsection{Interview}
Um nicht nur quantitative Daten zur Usability-Einschätzung der beiden Systeme zu sammeln, wurde zuletzt noch ein Interview über die zu Beginn der Studie gesetzten Marker geführt. Dazu ging der Versuchsleiter gemeinsam mit der Versuchsperson durch die während der Explorationsphase entstandene Bildschirmaufnahme. Bei jedem Marker wurde die Versuchsperson zu ihren dazugehörigen Gedanken, Gefühlen, ihrer Motivation und den dahinterliegenden Bedürfnissen befragt, während der Protokollant dies mitschrieb.

Das Interview führten wir am Ende der Studie durch, um eine Beeinflussung der restlichen Bewertungen, welche möglicherweise durch die bewusste Reflektion und die Interaktion mit dem Versuchsleiter entstehen könnten, auszuschließen.