\subsection{Teilnehmer}
Insgesamt nahmen an der Studie 28 Versuchspersonen teil, davon waren 16 weiblich und zwölf männlich. Das durchschnittliche Alter in Jahren betrug \textit{MW}~=~20.93 mit einer Standardabweichung von \textit{SD}~=~1.84.

Alle Teilnehmer waren zum Zeitpunkt der Studie Studenten, davon der Großteil entweder im Fachbereich Mensch-Computer-Systeme (\textit{n}~=~11) oder Medienkommunikation (\textit{n}~=~11). Die übrigen sechs Versuchspersonen kamen aus anderen Fachbereichen, wie Luft-und Raumfahrtinformatik, Mathematik, Mathematische Physik oder Wirtschaftswissenschaften. Der höchste Bildungsabschluss der Teilnehmer war vorwiegend die Allgemeine Hochschulreife (\textit{n}~=~25), einige erlangten bereits einen Hochschulabschluss (\textit{n}~=~3).

Die Entscheidung, ausschließlich Studenten als Versuchspersonen auszuwählen, wurde vor allem durch zwei Faktoren beeinflusst. Auf der einen Seite bietet die Wahl von Studenten durch die bestehende Infrastruktur ein attraktives Kosten-Nutzen-Verhältnis.
Auf der anderen Seite sind Projektarbeiten nach \cite{gensch2003bachelor} Teil von mehr als 75\% der  Bachelorstudiengänge an bayrischen Universitäten, sodass nicht nur Betriebe sondern auch Studenten vom Einsatz der Projektmanagementtools profitieren können.

Vor der Studie hatten nach eigener Angabe 78.57\% der Teilnehmer (\textit{n}~=~22) Trello und 100\% der Teilnehmer Zenkit noch nie genutzt. Zwei Probanden gaben an Trello zwischen ein und vier mal benutzt zu haben, vier Probanden gaben an Trello bereits mehr als fünf mal benutzt zu haben.