\newpage
\section{Zusammenfassung}
Dieser Bericht dokumentiert die Untersuchung und den Vergleich der beiden Projektmanagementtools \cite{trello} und \cite{zenkit} in Bezug auf deren Usability. Um zu überprüfen, welches der beiden Systeme dabei besser abschneidet, wurde zunächst eine Heuristische Evaluation durchgeführt und diese anschließend um eine Usability-Studie ergänzt.

Die von fünf Evaluatoren durchgeführte Heuristische Evaluation orienterte sich an den zehn Heuristiken von \cite{nielsen199510} und führte zu 44 entdeckten Problemen. Obwohl bei Trello dabei weniger Probleme entdeckt wurden als bei Zenkit bewahrheitete sich die Erwartung von dessen Überlegenheit nur bedingt, da der durchschnittliche Problemgrad den von Zenkit überstieg.

Bei der Usability-Studie sollten die Versuchspersonen beide Systeme im Rahmen einer Explorationsphase mithilfe von speziellen Markern bewerten, zu welchen später ein Interview folgte. Bei den so gewonnenen qualitativen Daten ergab sich eine sehr ähnliche Verteilung von Anmerkungen und Einschätzungen zwischen Trello und Zenkit.
Im Anschluss sollte jeweils eine Reihe von Aufgaben mit beiden Systemen bearbeitet und anschließend die Erfahrungen in Fragebögen festgehalten werden, u.a. dem NASA TLX von \cite{hart1988development} und dem AttrakDiff von \cite{hassenzahl2003attrakdiff}. Es wurden jeweils zwei quantitative Variablen für alle drei Kriterien der Gebrauchstauglichkeit erhoben.

In den so gewonnenen empirischen Daten zeigte sich in keinem Fall ein signifikanter Unterschied zwischen den beiden Systemen.

\newpage