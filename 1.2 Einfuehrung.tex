\newpage
\section{Einführung}

% Relevanz des Themas; 
Der Wert von Zusammenarbeit und Kollaboration in Teams kann kaum unterschätzt werden. Als elementarer Baustein beinahe jeden Projekts sind Teams unverzichtbar, und zugleich bleibt die Kommunikation innerhalb der Gruppe oft eine Herausforderung, deren Bewältigung jedoch mithilfe von Technologie erleichtert werden kann. Die Bedeutung von Projektmanagementtools wurde durch einen Werbeslogan des Softwarekonzerns Attlassian treffend formuliert:

'Von Medizin und Raumfahrt über Katastrophenhilfe bis hin zum Pizzaservice helfen unsere Produkte Teams in aller Welt, die Menschheit durch Software voranzubringen.', \cite{atlassianslogan}.

% Produkte und Nutzungskontext; 
Trello und Zenkit sind zwei mögliche Kandidaten einer Vielzahl solcher Tools, die es ermöglichen sollen, die Zusammenarbeit zwischen Menschen effektiver, effizienter und zufriedenstellender zu gestalten. 
% Rechtfertigung des Vergleichs; 
Als uneingeschränkte kostenfreie Optionen bieten sich Trello und Zenkit für einen Vergleich besonders an, zusätzlich auch aufgrund der weitreichenden Popularität Trellos (\cite{blogtrello}) und Zenkits regionalem Ursprung (\cite{zenkitinterview}).

Der Fokus der beiden (und vielen anderen ähnlichen) Systemen ist die Hilfe bei der  Organisation von Projekten durch Aufgabenverwaltung. Wie auch bei Trello und Zenkit wird hierbei oft das japanische Prinzip 'Kanban' genutzt, welches aus Toyotas Produktionssystem stammt (\cite{sugimori1977toyota}) und ähnlich wie das Scrum-Prinzip ein wichtiger Teil des Projektmanagements, und speziell der agilen Softwareentwicklung, darstellt. 

Die Systeme bieten die Möglichkeit, die zu lösenden Aufgaben auf einem sogenannten 'Board'\footnote{Anmerkung: Im Rahmen dieses Berichts wird für Konsistenz die Oberfläche, mit dem ein Projekt organisiert wird, mit dem Begriff 'Board' bezeichnet, obwohl diese bei Zenkit stattdessen als 'Collection' bezeichnet wird. Der Begriff 'Board' ist unserer Einschätzung nach geläufiger, u.a. vom Nutzungskontext der Softwareentwicklung ausgehend.}
in Listen zu organisieren, wo sie mit umfangreichen Eigenschaften ausgestattet werden können, u.a. Checklisten, Zuständigkeiten oder Fristen. Enorm wichtig für Produktivität und Kollaboration ist dabei, dass alle Mitglieder des Teams online und in Echtzeit auf die selben Aufgaben innerhalb eines Boards zugreifen können.

%  Evaluationsziele; Bezug zu anderen Arbeiten
Ziel dieser Studie ist die Beantwortung der Frage, welches der beiden Tools die bessere Usability vorweisen kann. Schlussendlich hängt davon die Qualität der Interaktion der User mit dem System ab, und somit auch die Interaktion und Kommunikation der Projektteilnehmer untereinander - ein kritischer Faktor für den Erfolg eines Projektes (\cite{cervone2014effective}).

% Einschränkungen und Besonderheiten
Bedingt durch den Rahmen der Entstehung dieser Studie beschränken wir uns auf diese beiden Systeme, die wir jedoch, durch die zusätzliche Nutzung empirischer Methoden, umfassender untersuchen als beispielsweise eine ähnliche Untersuchung von \cite{cicibas2010comparison}.